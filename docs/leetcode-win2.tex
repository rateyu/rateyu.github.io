% Created 2025-02-02 Sun 23:19
% Intended LaTeX compiler: pdflatex
\documentclass[11pt]{article}
\usepackage[utf8]{inputenc}
\usepackage[T1]{fontenc}
\usepackage{graphicx}
\usepackage{longtable}
\usepackage{wrapfig}
\usepackage{rotating}
\usepackage[normalem]{ulem}
\usepackage{amsmath}
\usepackage{amssymb}
\usepackage{capt-of}
\usepackage{hyperref}
\usepackage{fontspec}          % 使用 fontspec 支持现代字体
\usepackage{xeCJK}            % 使用 xeCJK 支持中文
\setCJKmainfont{PingFang} % 设置中文字体为 WenQuanYi Zen Hei
\setmainfont{DejaVu Sans}     % 设置英文字体为 DejaVu Sans
\usepackage{geometry}         % 设置页面尺寸
\geometry{a5paper}            % 设置A5纸张
\pdfpagewidth=210mm           % 设置PDF宽度为210mm
\pdfpageheight=148mm          % 设置PDF高度为148mm
\author{myu}
\date{\today}
\title{}
\hypersetup{
 pdfauthor={myu},
 pdftitle={},
 pdfkeywords={},
 pdfsubject={},
 pdfcreator={Emacs 28.2 (Org mode 9.5.5)}, 
 pdflang={English}}
\begin{document}

\tableofcontents


\section{算法53 最大子数组和}
\label{sec:org96fde17}
\begin{verbatim}
public int maxSubArray(int[] nums) {
    int max = nums[0];
    for(int i = 0; i < nums.length; i++) {
	int tmpMax = nums[i];
	if(tmpMax > max) {
	    max = tmpMax;
	}
	for(int j = i + 1; j < nums.length; j++) {
	    tmpMax += nums[j];
	    if(tmpMax > max) {
		max = tmpMax;
	    }
	}
    }
    System.out.println("max--" + max);
    return max;
}
\end{verbatim}
\end{document}